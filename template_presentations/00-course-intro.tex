
\documentclass{beamer}

\usepackage[utf8]{inputenc}
\usepackage[T1]{fontenc}
\usepackage[portuguese]{babel}
\usepackage[printwatermark]{xwatermark}
\usepackage{background}
\usepackage{listings}

\lstset{
    basicstyle=\footnotesize,
    numbers=left,
    numberstyle=\footnotesize,
    captionpos=b,
    tabsize=2
}

\backgroundsetup{
    placement=center,
    scale=1.15,
    contents={\url{https://www.livredigital.com}},
    color=black!40,
    angle=25,
    opacity=0.25
}

\setbeamertemplate{background}{\BgMaterial}

\usetheme{Warsaw}
\setbeamertemplate{headline}{}

% Title of your presentaiton
\title{Treinamento Online}

% A subtitle of your presentation
\subtitle{Como são realizados os cursos online da Livre Digital}

\author{Livre Digital}

\institute
{
  Nosso site: \url{http://www.livredigital.com}\\
  Socials: \href{https://www.twitter.com/LivreDigitalCom}{@LivreDigitalCom} | \href{https://www.github.com/livredigital}{Github} | \href{https://www.linkedin.com/company/livredigital}{LinkedIn}
}

% Last Revision
\date{Última Revisão: Maio, 2018.}

\begin{document}

% Make title
\begin{frame}
  \titlepage
\end{frame}

% License of document
\begin{frame}{Licença do Documento}
  Livre Digital, todos os direitos reservados \copyright{} 2018.\\
  \textbf{Licença:} Creative Commons Attribution - Share ALike 4.0\\
  \small{\url{https://creativecommons.org/licenses/by-sa/4.0/legalcode}}\\[1cm]
  Você tem permissão para:
  \begin{itemize}
      \item copiar, distribuir e apresentar,
      \item modificar este documento,
      \item usar este documento comercialmente;
  \end{itemize}
  Para isto você DEVE:
  \begin{itemize}
      \item Dar crédito ao autor original. \textbf{Atribution}.
      \item Utilizar a mesma licença deste documento. \textbf{Share Alike}.
  \end{itemize}
\end{frame}

% About Livre Digital
\begin{frame}{Livre Digital}
  \begin{minipage}{0.25\textwidth}
    \includegraphics[width=\linewidth]{imgs/logo.png}\\
    %\begin{center}
      \small{Startup digital.}
    %\end{center}
  \end{minipage}
  \begin{minipage}{0.7\textwidth}
    Livre Digital\\
    \small{\emph{Focando em soluções de código aberto}}\\
    \url{http://www.livredigital.com}
    \begin{itemize}
        \item Experiência em código aberto e software livre;
        \item Atividades: Desenvolvimento, treinamento, consultoria e suporte técnico.
    \end{itemize}
  \end{minipage}

  \\[0.65cm]

  \begin{tabular}{r l}
  Documentos & \url{https://www.livredigital.com/docs}\\
  Treinamentos & \url{https://www.livredigital.com/training}\\
  Softwares & \url{https://www.livredigital.com/softwares}\\[1cm]
  Socials & \href{https://www.twitter.com/LivreDigitalCom}{@LivreDigitalCom} | \href{https://www.github.com/livredigital}{Github} | \href{https://www.linkedin.com/company/livredigital}{LinkedIn}
  \end{tabular}
\end{frame}

\begin{frame}{Agenda}
  \linespread{2.5}
  \tableofcontents
\end{frame}

\subsection[]{Curso online}

\begin{frame}{Curso Online}
  Na \textbf{Livre Digital} nós realizamos alguns treinamentos in loco mas em sua
  maioria produzimos e realizamos conteúdos online. Estes treinamentos visam dar
  ao aluno a oportunidade de escolher os melhores horários para acompanhar as aulas,
  realizar os exercícios e as avaliações dos módulos.

  \\~\

  Os cursos online podem ser acessados a qualquer momento pelo no nosso website
  ou baixados para o computador e/ou celular do aluno, geralmente disponibilizamos
  todo o conteúdo escrito em formato de livro / e-book, há depender se ele é pago
  ou gratuito.

  \\~\

  Todos os nossos cursos são constantemente revisados e melhorados por nossa equipe,
  alunos que compraram cursos pagos tem sempre em nosso site o conteúdo atualizado
  para ser baixado quando necessário.
\end{frame}

\subsection[]{Recursos do Treinamento}

\begin{frame}{Recursos do Treinamento}
  Para fornecer ao aluno um melhor aproveitamento, nossos cursos em sua maioria
  disponibilizam os recursos listados abaixo:

  \begin{itemize}
    \item Apresentações de cada módulo.
    \item Documentação extensa sobre o assunto.
    \item Video aula com orientação em português e legendado.
    \item Lista de exercícios para melhor aprendizado.
    \item Avaliações para verificar o entendimento do aluno.
    \item Livro / e-book sobre o tema.
    \item Certificado digital para curriculum.
    \item Suporte via chat / e-mail.
  \end{itemize}

  * Alguns destes recursos podem não estar disponíveis em cursos que fornecemos
  gratuitamente para iniciantes.
\end{frame}

\subsection[]{Exercícios e Avaliações}

\begin{frame}{Exercícios e Avaliações}
  Em nossos treinamentos acreditamos que a realização de \textbf{exercícios} é fundamental
  para o aprendizado, por isso cada um de nossos módulos independente do curso ser
  gratuito ou pago contém uma série de exercícios.

  \\~\

  Obviamente cursos pagos contém exercícios mais complexos e bem elaborados além
  da resolução dos mesmos disponível para os alunos em nosso site. O aluno pode
  concluir os exercícios e nos enviar para que o nosso sistema automaticamente
  valide se a resposta está correta e ele possa ganhar pontos.

  \\~\

  Em cursos pagos e alguns cursos gratuitos temos também \textbf{avaliações} contendo uma
  série \textit{de 6 (seis) até 12 (doze) questões} em média para que o aluno que desejar
  o \textbf{certificado digital} de conclusão do curso possa ser avaliado no seu entendimento
  geral.
\end{frame}

\subsection[]{Conclusão e Certificados}

\begin{frame}[fragile]{Conclusão e Certificados}
  O aluno que acessa todos os módulos do curso pode optar por simplesmente fechar
  seu navegador e ir direto para prática no seu projeto pessoal ou na sua empresa,
  mas também caso deseje se esforçar um pouco mais pode concluir a \textit{pontuação mínima
  necessária} para se obter um \textbf{certificado digital}.

  \\~\

  Este certificado digital pode ser utilizado para colocar em sua página no LinkedIn
  ou até mesmo como comprovação de conhecimento e realização de um curso sobre
  determinado assunto no seu curriculum vitae.

  \\~\

  Realizamos a validação do seu certificado em nosso website. Isto quer dizer que
  o seu certificado fica sempre visível para quem tem o link, podendo servir de
  comprovação para o seu cliente ou empregador de que foi feito o curso e obtido
  o certificado.

\end{frame}

\begin{frame}{}
  \begin{center}
    \vfill
    \Huge{Então, o que está esperando!?}
    \vfill
    \Huge{Acesse a próxima lição!}
    \vfill
  \end{center}
\end{frame}

\end{document}
